% LaTeX file for resume / cv A
% This file uses the resume document class (res.cls)


\documentclass[line]{res} 
\usepackage{color}
\usepackage{randtext}
\usepackage{multicol}
\usepackage{hyperref}

\usepackage{enumitem}

%\setlength{\textheight}{9.5in} % increase text height to fit on 1-page 

\newcounter{problemcounter}
\newcommand{\paper}{%
  \stepcounter{problemcounter}%
  \textbf{\theproblemcounter}}

\begin{document}

\name{Thomas Gilray}  

\address{\bf gilray.org
\ \ \ \ \ \ \ \ \randomize{thomas.gilray@wsu.edu}}
                                  
\begin{resume}        

\vspace{-0.1in}  
I am currently an Associate Professor in Computer Science at Washington State University, and was previously Assistant Professor at the University of Alabama at Birmingham and a Victor Basili Fellow at the University of Maryland, at College Park.

\section{\large Research Summary} \vspace{0.1cm}   
My research interests center around the design and (scalable) implementation of high-level programming languages and systems for reasoning automatically about programs. I have contributed to the design of tunable whole-program control-flow analyses, information-flow analyses, contract verification systems, and have invented novel Datalog-based languages for specifying these analyses declaratively and implementing them efficiently. I develop high-performance-computing techniques for accelerating data-parallel relational algebra and sparse linear algebra on clusters and GPUs.

\section{\large Education}          
    \vspace{-0.1cm}	
    \begin{tabbing}
    \hspace{2.25in}\= \hspace{2.25in}\= \kill % set up two tab positions
    {\bf University of Utah} \>Ph.D.     \>2017
    \end{tabbing}\vspace{-0.5cm}
    Developed a unified methodology for polyvariant (\textit{e.g.}, flow/call/arg/obj sensitive) program analysis.% and applied the approach to produce a 
    %specific adaptive style of polyvariance, obtaining a form of perfect precision at no asymptotic cost to analysis complexity. 
    
    I applied this framework to produce a self-reflective form of polyvariance for continuations that avoids all return-flow conflation of values (a long-standing problem for control-flow analyses), guaranteeing ideal stack precision at no cost to average or worst-case model complexity, and at in terms of human labor to implement ($\sim{}\!\!1$LOC in proposed framework).
    %
    Proved that the precision is equal to an incomputable analysis with an unbounded stack and mechanically verified the proof using the Coq proof assistant.
    %
    My dissertation is titled \textit{``Introspective Polyvariance for Control-Flow Analyses''}.
    %
   \vspace{-0.2cm} 
    \begin{tabbing}
    \hspace{2.25in}\= \hspace{2.25in}\= \kill % set up two tab positions
    {\bf University of Utah} \>M.S.     \>2012
    \end{tabbing}\vspace{-0.5cm}
%    Worked as a department TA and then as an RA in Matt Might's U-combinator lab.
    %
    \vspace{-0.2cm} 
    \begin{tabbing}
    \hspace{2.25in}\= \hspace{2.25in}\= \kill % set up two tab positions
    {\bf University of Oregon} \>B.S.     \>2010
    \end{tabbing}\vspace{-0.5cm}
%    Received a B.S. in Computer Science, Spring 2010, with a minor in Business Administration.
    %
%    Focused undergraduate studies on programming languages, interpreters, compilers, and data structures.

\vspace{0.1in}
\section{\large Employment}
   \vspace{-0.05in}  
   \begin{tabbing}
    \hspace{2.25in}\= \hspace{2.25in}\= \kill % set up two tab positions
    {\bf Associate Professor} \>Washington State University  \>2024 - \textit{Present}
   \end{tabbing}\vspace{-0.5cm}      % suppress blank line after tabbing
   Working on research into scalable, tunable program analysis, program verification, logic solvers on high performance computers and clusters, and linguistic mechanisms for enforcing correctness, security/privacy, and termination properties of software; regularly teaching undergraduate and graduate-level classes in compilers and programming languages. Won an ARPA-H subaward to investigate private and secure querying for EHR systems, contracting with with Galois, inc.
   %
   \vspace{-0.2cm}  
   \begin{tabbing}
    \hspace{2.25in}\= \hspace{2.25in}\= \kill % set up two tab positions
    {\bf Assistant Professor} \>U. of Alabama, Birmingham  \>2018 - 2024
   \end{tabbing}\vspace{-0.5cm}      % suppress blank line after tabbing
   Developed my research program in high-performance reasoning and program analysis; regularly taught undergraduate and graduate-level classes in automata theory, programming languages, and automated reasoning. Won NSF PPoSS Large, NSF PPoSS Planning, and DARPA VSPELLS grants, bringing a total of \$3.12M in new research money to UAB.
   %
   \vspace{-0.2cm}  
   \begin{tabbing}
    \hspace{2.25in}\= \hspace{2.25in}\= \kill % set up two tab positions
    {\bf Victor Basili Fellow} \>U. of Maryland, College Park  \>2016 - 2018
   \end{tabbing}\vspace{-0.5cm}      % suppress blank line after tabbing
   Joined UMD's PLUM lab with Michael W. Hicks, Jeffery Foster, and David Van Horn; worked on various collaborative projects including: soft contract verification, approximating permission-use provenance in Android, accelerating flow analyses in Datalog, and verification of faceted programs, among others. A departmental fellowship granted me great freedom to pursue long-term research.
   %
   \vspace{-0.2cm}  
   \begin{tabbing}
    \hspace{2.25in}\= \hspace{2.25in}\= \kill % set up two tab positions
    {\bf Instructor} \>U. of Maryland, College Park  \>2017 - 2018
   \end{tabbing}\vspace{-0.5cm}      % suppress blank line after tabbing
   While at UMD, I taught a section of the Intro to Programming Languages course and developed a new Compilers course in which my students built an R\textsuperscript{7}RS Scheme Compiler from scratch. 
   %
   \vspace{-0.2cm}
   \newpage
   \begin{tabbing}
    \hspace{2.25in}\= \hspace{2.25in}\= \kill % set up two tab positions
    {\bf Analysis Developer} \>HP, inc. (was: Fortify, inc.)     \>2013 - 2015
   \end{tabbing}\vspace{-0.5cm}      % suppress blank line after tabbing
   Following a graduate internship in Summer 2013, I was hired to work remotely, concurrent with my work in the U-combinator lab.
   I implemented a new inter-procedural must-alias analysis based on the theory of \emph{abstract counting} to supplement existing analyses in the Fortify source-code analyzer (SCA).
   I successfully addressed various bug reports and lead an architectural review of SCA's ``phase 0'' preprocessing code which performs SSA conversion and intra-procedural analysis.
    %
   \vspace{-0.2cm}  
   \begin{tabbing}
    \hspace{2.25in}\= \hspace{2.25in}\= \kill % set up two tab positions
    {\bf Research Assistant} \>U. of Utah     \>2011 - 2016
   \end{tabbing}\vspace{-0.5cm}      % suppress blank line after tabbing
    I studied polyvariant program analysis in Matt Might's U-combinator lab, implementing analyses as
    traditional operational semantics (abstract interpreters) and as linear algebra, to exploit inherent parallelism, and designing a new dynamic sparse-matrix format for SIMD hardware (such as GPUs).
    %
    I worked on Android analysis and timed challenge events for DARPA's APAC program. 
    %
   \vspace{-0.2cm}  
   \begin{tabbing}
    \hspace{2.25in}\= \hspace{2.25in}\= \kill % set up two tab positions
    {\bf Teaching Assistant} \>U. of Utah     \>2010 \& 2012
   \end{tabbing}\vspace{-0.5cm}      % suppress blank line after tabbing
    I taught three sections of 200-level \textit{Discrete Structures} as a teaching assistant and also helped to develop projects and tests for \textit{Compilers} and \textit{Advanced Compilers}, taught by my advisor.
    %
   \vspace{-0.2cm}  
   \begin{tabbing}
    \hspace{2.25in}\= \hspace{2.25in}\= \kill % set up two tab positions
    {\bf Co-founder, Developer} \>Peculi, llc.    \> 2006 - 2009
   \end{tabbing}\vspace{-0.5cm}      % suppress blank line after tabbing
   Helped to raise \$3,000 in start-up funds to build an art-sharing platform. Built out a high-performance backend and AJAST-style
   load-balancing to handle requests from a web-gallery front-end in Javascript. The site was home to more than $14,000$ works of art contributed
   by roughly $800$ active users. Wrote a message-board engine with social-media features. Wrote a secure CAPTCHA and various image processing
   scripts. Served as system administrator for the site's lifetime.
    %
   \vspace{-0.2cm}  
   \begin{tabbing}
    \hspace{2.25in}\= \hspace{2.25in}\= \kill % set up two tab positions
    {\bf Intern, Freelance, \textit{etc.}} \>Various     \> 2005 - 2010
   \end{tabbing}\vspace{-0.5cm}      % suppress blank line after tabbing
   Before my graduation from UO in 2010, I worked a variety of jobs: for HP, inc. developing tooling and a tutorial for using 3DSMAX to visualize Autodesk Inventor models; for Prof. Ostroverkhova (at OSU) developing a Latex-based CMS for PhD students in Physics; for fourteen companies or individuals (some repeatedly) as a freelance developer, mostly creating dynamic websites, implementing CSS designs, and customizing CMS deployments; for a startup VizMe, inc. co-founded by Prof. Eric Wills (from UO), setting up a Solr-based search engine in Java and writing Python scripts to manage MySQL databases, site-localization, and image analyses.

   
\section{\large Awards \& Recognition} \vspace{0.2in}
\begin{itemize}
\item Won ISC Hans Meuer Best Paper award (2020).
\item Featured by ALCF: collaboration with Sidharth Kumar advancing scalable MPI-based relational algebra was featured by ALCF's yearly \emph{Science Report} magazine as a research highlight growing from our 2019 D.D. grant of hours on ALCF's Theta supercomputer. \url{https://www.alcf.anl.gov/sites/default/files/2021-04/ALCF_2020ScienceReport.pdf} (page 37) (2020).
\item Invited to the Journal of Functional Programming (2019).
\item Won HiPC Best Paper award (2019).
\item DOE Directors Direction (D.D.) Grant of 2M hours on ALCF's Theta (2019).
\item Invited to the Journal of Functional Programming (2018).
\item Won PRACE ISC best paper award (2016).
\item Victor Basili Fellowship at the University of Maryland, College Park (2016).
\item Won TFP best student paper award (2013).
\end{itemize}


    
\section{\large Teaching} \vspace{0.2in}
    
\textbf{(WSU) CptS 580: Advanced Programming Languages} (Spring 2025)

I designed from scratch, and taught, an advanced programming-language theory course for graduate students at WSU, Spring 2025. The course focused on programming language theory, the lambda calculus, different styles of formal semantics, especially operational semantics, interpreters, type systems and let polymorphism, and several styles of program analysis.
    
\textbf{(WSU) CptS 452: Compilers} (Fall 2024)

I designed from scratch, and taught, a course on compiler implementation at WSU. The class was based around a central project, composed of managable pieces, where students build their own Scheme-to-C compiler from scratch. The course covered standard intermediate representations, implementation of various constructs via first-class control, implementation of lambdas as closures, and a runtime that integrates the Boehm garbage collector. 

\textbf{(UAB) CS 660/760: Artificial Intelligence} (Fall 2019--2021, 2023)

I designed and taught a course that focuses on symbolic AI, planning, logic, and automated reasoning (UAB had six other courses focusing on Machine Learning, which is outside my primary areas of expertise). We focused on problem solving by search, adversarial search and games, automated constraint solving, SAT solving, and logic. %Students complete group projects focusing on A*, two-player Reversi, and Sudoku solving and complete an individual term project. As my own research in Datalog falls within the scope of this course, I spend at least a week near the end of the term focusing on logic programming and HornSAT/Datalog after the students have implemented DPLL.

\textbf{(UAB) CS 401/501: Programming Languages} (Spring 2019--2024)

I designed and taught a Programming Languages course for undergraduates at UAB. The course surveys a variety of languages features and paradigms, focusing especially on introducing functional and declarative programming, language design, type systems, semantics, and interpreters. The class is programming intensive, with a series of six projects and biweekly exercises, and includes a modest focus on implementation strategies. Students build a Church compiler, small-step and big-step interpreters, and a purely functional implementation of PageRank, among other unique projects.

\textbf{(UAB) CS 350/550: Automata and Formal Languages} (Fall 2018--2023)

I improved UAB's course on automata, formal languages, and the theory of computation by adding two substantial coding projects to supplement the class' theoretical focus. I added a project where students build an automata library to convert a given regular expression to an NFA, convert this to a DFA, and then minimize the DFA. This library can match strings using any intermediate model of regular languages and can visualize any NFA/DFA using Graphviz so students can visually inspect the NFAs and DFAs they are building. I have also increased the class' focus on parsing and added a project where students write their own recursive-descent parser for a small language.

\textbf{(UMD) CMSC 330: Intro to Programming Languages} (Spring 2018)

I taught a spring section of CMSC 330, UMD's core Programming Languages course which covers a series of languages and paradigms (using Ruby, OCaml, Rust, Prolog) and core PL theory on lambda calculus, semantics, and type systems. 

\textbf{(UMD) CMSC 430: Compilers} (Fall 2017)

I designed and taught an elective course on compiler implementation at UMD. The class was based around a central project, composed of bite-sized chunks, where students build their own Scheme-to-LLVM-IR compiler, from scratch. 

\section{\large Doctoral Theses} \vspace{0.3cm}
\paper. \textit{Introspective Polyvariance for Control-Flow Analyses.}
\textbf{Thomas Gilray}.
University of Utah.
\\(U of U) 2016.
\\ \vspace{-0.1cm}\\

\vspace{-0.4cm}

\section{\large Journal Papers (Refereed)} \vspace{0.3cm}
\indent $\dagger$ Indicates a student advisee and $\ddagger$ indicates a student mentee.

\paper. \textit{Abstract Allocation as a Unified Approach to Polyvariance in Control-flow Analyses.}
\textbf{Thomas Gilray}, Michael D. Adams, and Matthew Might.
Journal of Functional Programming.
\\(JFP) 2018. 
\\\textbf{Invited to the Journal of Functional Programming.} \\ \vspace{-0.1cm}\\

\vspace{-0.4cm} 

\section{\large Conference Papers (In Submission)}
%\indent $\dagger$ Indicates a student advisee and $\ddagger$ indicates a student mentee.

\vspace{0.2cm}
\paper. \textit{Parallel Prefix Joins for Linear Recursive Datalog Rules.}
Akash Rao$\dagger$, \textbf{Thomas Gilray}, and Ananth Kalyanaraman.
International Conference on Supercomputing.
\\(ICS) Feb 2025. (In submission)
\\ \vspace{-0.1cm}\\
\paper. \textit{Multi-Node Mutli-GPU Datalog.}
Ahmedur Rahman Shovon$\ddagger$, Yihao Sun$\ddagger$, \textbf{Thomas Gilray}, Sidharth Kumar, and Kristopher Micinski.
International Conference on Supercomputing.
\\(ICS) Feb 2025. (In submission)
\\ \vspace{-0.1cm}\\

\vspace{-0.4cm}


\section{\large Conference Papers (Refereed)}
\vspace{0.1cm}
\paper. \textit{Datalog with First-class Facts.}
\textbf{Thomas Gilray}, Arash Sahebolamri, Yihao Sun, Sowmith Kunapaneni, Sidharth Kumar, and Kristopher Micinski.
International Conference on Very Large Data Bases.
\\(VLDB) Sep 2025. 
\\ \vspace{-0.1cm}\\
\paper.$\dagger$ \textit{Optimizing Datalog for the GPU.}
Yihao Sun, Ahmedur Rahman Shovon, \textbf{Thomas Gilray}, Sidharth Kumar, and Kristopher Micinski.
International Conference on Architectural Support for Programming Languages and Operating Systems.
\\(ASPLOS---12.7\% acceptance) Mar 2025. 
\\ \vspace{-0.1cm}\\
\paper.$\dagger$ \textit{Column-Oriented Datalog on the GPU.}
Yihao Sun, Sidharth Kumar, \textbf{Thomas Gilray}, and Kristopher Micinski.
AAAI Conference on Artificial Intelligence.
\\(AAAI---23.4\% acceptance) Feb 2025. 
\\ \vspace{-0.1cm}\\
\paper.$\dagger$ \textit{Analysis of MPI Communication Time for Distribution of Repartitioned Data.}
John-Paul Robinson, Ke Fan, Sidharth Kumar, Steve Petruzza, and \textbf{Thomas Gilray}.
Practice and Experience in Advanced Research Computing.
\\(PEARC) Jul 2024. 
\\ \vspace{-0.1cm}\\
\paper.$\dagger$ \textit{Configurable Algorithms for All-to-all Collectives.}
Ke Fan, Steve Petruzza, \textbf{Thomas Gilray}, and Sidharth Kumar.
ISC High Performance.
\\(ISC---29\% acceptance) May 2024. 
\\ \vspace{-0.1cm}\\
\paper.$\dagger$ \textit{Communication-Avoiding Recursive Aggregation.}
Yihao Sun, Sidharth Kumar, \textbf{Thomas Gilray}, and Kristopher Micinski.
IEEE International Conference on Cluster Computing.
\\(CLUSTER) Nov 2023. 
\\ \vspace{-0.1cm}\\
\paper.$\dagger$ \textit{Towards Iterative Relational Algebra on the GPU.}
Ahmedur Rahman Shovon, \textbf{Thomas Gilray}, Kristopher Micinski, and Sidharth Kumar.
USENIX ATC.
\\(USENIX ATC) Jul 2023. 
\\ \vspace{-0.1cm}\\
\paper.$\dagger$ \textit{Optimizing the Bruck Algorithm for Non-uniform All-to-all Communication.}
Ke Fan, \textbf{Thomas Gilray}, Valerio Pascucci, Xuan Huang, Kristopher Micinski, and Sidharth Kumar.
International ACM Symposium on High-Performance Parallel and Distributed Computing.
\\(HPDC---19\% acceptance) Jun 2022. 
\\ \vspace{-0.1cm}\\
\paper.$\dagger$ \textit{Seamless Deductive Inference Via Macros.}
Arash Sahebolamri, \textbf{Thomas Gilray}, and Kristopher Micinski.
International Conference on Compiler Construction.
\\(CC) Apr 2022. 
\\ \vspace{-0.1cm}\\
\paper.$\dagger$ \textit{Load-balancing Parallel I/O of Compressed Hierarchical Layouts.}
Ke Fan, Duong Hoang, Steve Petruzza, \textbf{Thomas Gilray}, Valerio Pascucci, and Sidharth Kumar.
International Conference on High Performance Computing, Data, and Analytics.
\\(HiPC) Dec 2021. 
\\ \vspace{-0.1cm}\\
\paper. \textit{Compiling Data-parallel Datalog.}
\textbf{Thomas Gilray}, Sidharth Kumar, and Kristopher Micinski.
International Conference on Compiler Construction.
\\(CC) Mar 2021. 
\\ \vspace{-0.1cm}\\
\paper. \textit{Load-balancing Parallel Relational Algebra.}
Sidharth Kumar and \textbf{Thomas Gilray}.
ISC High Performance.
\\(ISC---31\% acceptance) Jun 2020. 
\\\textbf{Won ISC Hans Meuer Best Paper award.} \\ \vspace{-0.1cm}\\
\paper. \textit{Abstracting Faceted Execution.}
Kristopher Micinski, David Darais, and \textbf{Thomas Gilray}.
IEEE Computer Security Foundations Symposium.
\\(CSF) Jun 2020. 
\\ \vspace{-0.1cm}\\
\paper. \textit{Distributed Relational Algebra at Scale.}
Sidharth Kumar and \textbf{Thomas Gilray}.
International Conference on High Performance Computing, Data, and Analytics.
\\(HiPC---23\% acceptance) Dec 2019. 
\\\textbf{Won HiPC Best Paper award.} \\ \vspace{-0.1cm}\\
\paper. \textit{Size-Change Termination as a Contract.}
Phúc C. Nguyễn, \textbf{Thomas Gilray}, Sam Tobin-Hochstadt, and David Van Horn.
Programming Language Design and Implementation.
\\(PLDI---27\% acceptance) Jun 2019. 
\\\textbf{Invited to the Journal of Functional Programming.} \\ \vspace{-0.1cm}\\
\paper. \textit{Soft Contract Verification for Higher-order Stateful Programs.}
Phúc C. Nguyễn, \textbf{Thomas Gilray}, Sam Tobin-Hochstadt, and David Van Horn.
Symposium on Principles of Programming Languages.
\\(POPL---23\% acceptance) Jan 2018. 
\\ \vspace{-0.1cm}\\
\paper. \textit{User Comfort with Android Background Resource Accesses in Different Contexts.}
Daniel Votipka, Seth M. Rabin, Kristopher Micinski, \textbf{Thomas Gilray}, Michelle L. Mazurek, and Jeffrey S. Foster.
Symposium on Usable Privacy and Security.
\\(SOUPS---21\% acceptance) 2018. 
\\ \vspace{-0.1cm}\\
\paper. \textit{Allocation Characterizes Polyvariance.}
\textbf{Thomas Gilray}, Michael D. Adams, and Matthew Might.
International Conference on Functional Programming.
\\(ICFP---31\% acceptance) Sep 2016. 
\\ \vspace{-0.1cm}\\
\paper. \textit{Pushdown Control-Flow Analysis for Free.}
\textbf{Thomas Gilray}, Steven Lyde, Michael D. Adams, Matthew Might, and David Van Horn.
Symposium on Principles of Programming Languages.
\\(POPL---23\% acceptance) Jan 2016. 
\\ \vspace{-0.1cm}\\
\paper. \textit{Dynamic Sparse-Matrix Allocation on GPUs.}
James King, \textbf{Thomas Gilray}, Robert M. Kirby, and Matthew Might.
ISC High Performance.
\\(ISC) 2016. 
\\\textbf{Won PRACE ISC best paper award.} \\ \vspace{-0.1cm}\\
\paper. \textit{A Survey of Polyvariance in Abstract Interpretations.}
\textbf{Thomas Gilray} and Matthew Might.
Symposium on Trends in Functional Programming.
\\(TFP) May 2013. 
\\\textbf{Won TFP best student paper award.} \\ \vspace{-0.1cm}\\

\vspace{-0.4cm}

\section{\large Workshop Papers (Refereed)} \vspace{0.3cm}
\paper. \textit{A Visual Guide to MPI All-to-all.}
Nick Netterville, Ke Fan, Sidharth Kumar, and \textbf{Thomas Gilray}.
Workshop on Education for High Performance Computing.
\\(EduHiPC) Dec 2022. 
\\ \vspace{-0.1cm}\\
\paper. \textit{Accelerating Datalog Applications with cuDF.}
Ahmedur Rahman Shovon, Landon Richard Dyken, Oded Green, \textbf{Thomas Gilray}, and Sidharth Kumar.
Workshop on Irregular Applications: Architectures and Algorithms.
\\(IA3) Nov 2022. 
\\ \vspace{-0.1cm}\\
\paper. \textit{Exploring MPI Collective I/O and File-per-process I/O for Checkpointing a Logical Inference task..}
Ke Fan, Kristopher Micinski, \textbf{Thomas Gilray}, and Sidharth Kumar.
Workshop on High Performance Storage.
\\(HPS) May 2021. 
\\ \vspace{-0.1cm}\\
\paper. \textit{Symbolic Path Tracing to Find Android Permission-Use Triggers.}
Kristopher Micinski, \textbf{Thomas Gilray}, Daniel Votipka, Jeffrey S. Foster, and Michelle L. Mazurek.
Workshop on Binary Analysis Research.
\\(BAR) Jan 2019. 
\\ \vspace{-0.1cm}\\
\paper. \textit{Racets: Faceted Execution in Racket.}
Kristopher Micinski, Zhanpeng Wang, and \textbf{Thomas Gilray}.
Scheme Workshop.
\\(SW) Sep 2018. 
\\ \vspace{-0.1cm}\\
\paper. \textit{Toward Parallel CFA with Datalog, MPI, and CUDA.}
\textbf{Thomas Gilray} and Sidharth Kumar.
Scheme Workshop.
\\(SW) Sep 2017. 
\\ \vspace{-0.1cm}\\
\paper. \textit{A Linear Encoding of Pushdown Control-Flow Analysis.}
Steven Lyde, \textbf{Thomas Gilray}, and Matthew Might.
Scheme Workshop.
\\(SW) Nov 2014. 
\\ \vspace{-0.1cm}\\
\paper. \textit{Concrete and Abstract Interpretation: Better Together.}
Maria Jenkins, Leif Andersen, \textbf{Thomas Gilray}, and Matthew Might.
Scheme Workshop.
\\(SW) Nov 2014. 
\\ \vspace{-0.1cm}\\
\paper. \textit{Partitioning 0-CFA for the GPU.}
\textbf{Thomas Gilray} and Matthew Might.
Workshop on Functional and Constraint Logic Programming.
\\(WFLP) Aug 2014. 
\\ \vspace{-0.1cm}\\
\paper. \textit{A Unified Approach to Polyvariance in Abstract Interpretations.}
\textbf{Thomas Gilray} and Matthew Might.
Scheme Workshop.
\\(SW) Nov 2013. 
\\ \vspace{-0.1cm}\\
\paper. \textit{Sound and Precise Malware Analysis for Android via Pushdown Reachability and Entry-Point Saturation.}
Shuying Liang, Andrew W. Keep, Matthew Might, David Van Horn, Steven Lyde, \textbf{Thomas Gilray}, and Petey Aldous.
ACM CCS Workshop on Security and Privacy in Smartphones and Mobile Devices.
\\(SPSM) Nov 2013. 
\\ \vspace{-0.1cm}\\

\vspace{-0.4cm}

\section{\large Mentoring \& Advising}

\paragraph{MS/PhD Advisees (As Committee Chair)}
\begin{itemize}
\vspace{0.15cm}\item 2022--present Nick Netterville (MS)
\item 2022--present Ashraful Islam (PhD)
\item 2019--2020 Kyle Headley (PhD)---Left for personal reasons at the start of the pandemic.
\end{itemize}
\paragraph{MS/PhD Advisees (As Committee Member)}
\begin{itemize}
\vspace{0.15cm}\item 2021--present Akmedur Rahman Shovon (PhD)
\item 2020--present Yihao Sun (PhD)---At Syracuse University.
\item 2019--present Arash Sahebolamri (PhD)---At Syracuse University.
\item 2019--present Ke Fan (PhD)
\end{itemize}
\paragraph{Other Mentees}
\begin{itemize}
\vspace{0.15cm}\item 2022--present Landon Dyken (PhD)
\item 2022--present Akshar Patel (MS)
\item 2022--present Michael Gathara (MS)
\item 2021 Laura Thompson (BS)
\item 2018--2021 Clark Ren (MS)---Now at Amazon, inc.
\item 2016--2019 Phúc C. Nguyễn (PhD)---Now at Google, inc.
\item 2016--2018 David Darais (PhD)---Now at Galois, inc.
\item 2016--2018 Kristopher Micinski (PhD)---Now a faculty member at Syracuse University.
\item 2016--2017 Javran Cheng (MS)---Now at Google, inc.
\item 2014--2016 Guannan Wei (MS)---Now a PhD student at Purdue University.
\item 2013--2014 Maria Jenkins (BS)---Now at Pixio, inc.
\item 2011--2014 Leif Andersen (BS)---Now a PhD student at Northeastern University.
\end{itemize}


\section{\large Awarded Grant Proposals} \vspace{0.45cm}

\begin{itemize}
  \item
    ARPA-H: HealthyDocs: Secure Interoperable Health Data Management. 2025-2029. (co-PI, in collaboration with
    %Matthew Might, Taisa Kushner, David Darais, Gregory Forlenza, Paul Wadwa, Todd Alonso, Cari Berget
    the University of Colorado Barbara Davis Center, the University of Alabama at Birmingham Precision Medicine Institute, and Galois, inc.) \textbf{(Total: \$10M; My lab's share: \$1.29M)} 
  \item
    NSF: PPoSS: Large: A Full-stack Approach to Declarative Analytics at Scale. 2023-2028. (Lead PI, in collaboration with
    %Sidharth Kumar, Kristopher Micinski, Swarat Chaudhuri, Suren Byna, Ananth Kalyanaraman, and Matthew Might
    Ohio State University, the University of Texas at Austin, the University of Alabama at Birmingham, the University of Illinois Chicago, and Syracuse University) \textbf{(Total: \$5M; My lab's share: \$1.05M)} 
  \item
  NSF: PPoSS: Planning: A Full-stack Approach to Declarative Analytics at Scale. 2022-2023. (co-PI, with Syracuse University) \textbf{(Total: \$250,000; My lab's share: \$83,116)} 
  \item
  DARPA: VSPELLS: Promotion to Optimal Languages Yielding Modular Operator-driven Replacements and Programmatic Hooks (POLYMORPH). 2021-2025. (PI, via subcontract with Galois, inc) \textbf{(Total: \$8M; My lab's share: \$400,000)}
\end{itemize}

\section{\large Pending Grant Proposals} \vspace{0.45cm}

\begin{itemize}
\item DOE: SC ASCR: AI-Driven Autonomous Visualization and Data Management for Scientific Discovery on Supercomputers. 2025-2030. (Co-PI, in collaboration with Aric Hagberg, Pascal Grosset, George Stelle, and Terry Turton at LANL) \textbf{(Total: \$19.26M; My lab's share: \$749,999)}
\item NSF: CICI: Architecting Resilient and Robust Cyberinfrastructure for Secure and Scalable Data Analytics in the Cloud. 2025-2028. (Co-PI, in collaboration with Xu Lin at WSU and Sidharth Kumar and Anrin Chakraborti at UIC) \textbf{(Total: \$1.2M; My lab's share: \$300,000)}
\end{itemize}

\section{\large Grant Reviewing} \vspace{0.45cm}
\begin{itemize}
  \item NSF Panelist, SHF CS2 Program, 2024
  \item NSF Panelist, PPoSS Program, 2024
  \item NSF Panelist, PPoSS Program, 2021
\end{itemize}

\section{\large Journal Reviewing} \vspace{0.45cm}
\begin{itemize}
  \item Transactions on Programming Languages and Systems (TOPLAS) 2023.
  \item Transactions on Computational Biology and Bioinformatics (TCBB) 2019. 
\end{itemize}

\section{\large Conference Program Committees} \vspace{0.45cm}
\begin{itemize}
  \item International Conference on Software Engineering (ICSE) 2025.
  \item International Conference on Compiler Construction (CC) 2025.
  \item High-Performance Computing (HiPC) 2024.
  \item High-Performance Computing Student Research Symposium (HiPC SRS) 2023.
  \item Programming Language Design and Implementation Student Research Competition (PLDI SRC) 2022.
  \item Dynamic Languages Symposium (DLS) 2021.
  \item Symposium on Applied Computing (SAC) 2021.
  \item Principles of Programming Languages (POPL) 2020 (ERC).
  \item Scheme and Functional Programming Workshop (SW) 2019 (PC Chair).
  \item MiniKanren Workshop (MW) 2019.
  \item Principles of Programming Languages (POPL) 2019 (ERC).
  \item European Conference on Object-Oriented Programming (ECOOP) 2018 (AEC).
  \item Principles of Programming Languages (POPL) 2018 (ERC).
\end{itemize}

\section{\large Departmental (Non-thesis) Committees} \vspace{0.45cm}
\begin{itemize}
  \item Qualifying Exam Committee (Chair) for Anais Barja, WSU, 2025.
  \item Faculty Search Committee, WSU, 2024.
  \item Graduate Admissions Committee, UAB, 2023-2024.
  \item Qualifying Exam Committee for Guimu Guo, UAB, 2019.
  \item Curriculum Committee, UAB, 2018-2023.
\end{itemize}

\section{\large Community \& Outreach Activities} \vspace{0.45cm}
\begin{itemize}
  \item WSU Senior Design Poster Competition Judge, 2025.
  \item UAB High-School Programming Contest (HSPC) Organizer, 2024.
  \item High-Performance Computing Student Research Symposium (HiPC SRS) Mentor and Judge, 2023.
  \item UAB High-School Programming Contest (HSPC) Organizer, 2023.
  \item ABC's Talk of Alabama (TV Panel Discussion): \href{https://tinyurl.com/abc-talk-of-alabama-gilray}{\emph{Artificial Intelligence \& Your World: What is it, and where is it headed?}}, 2023.
  \item ABC's Talk of Alabama (TV Interview): \href{https://tinyurl.com/abc-talk-of-alabama-gilray}{\emph{What is Artificial Intelligence?}}, 2023.
  \item Programming Languages Mentoring Workshop (PLMW) Mentor and Panelist, \emph{``Formal Methods: Theory and Practice''}, 2022.
  \item UAB High-School Programming Contest (HSPC) Organizer, 2021.
  \item Organized and ran the Scheme and Functional Programming Workshop, 2019.
  \item Lead a student team in developing an automated grading system (\href{https://autograde.org}{autograde.org}), 2019.
  \item UAB High-School Programming Contest (HSPC) Organizer, 2019.  
  \item Faculty advisor to the UAB student ACM \& ACM-W, 2018-2024.
  \item ACM and SIGPLAN member since before 2013 or before.
\end{itemize}

\section{\large Invited Talks} \vspace{0.2in}
\begin{itemize}
\item Data-parallel Datalog with First-class Facts. Minnowbrook Logic-programming Seminar. Syracuse University. 2025.
\item Correctness and Verification using Software Contracts. Cyser Summer Workshop. Washington State University, 2025.
\item Challenges in High-performance Deductive Programming. AP2S: Automated Program and Proof Synthesis. Vancouver, BC. AAAI Bridge. 2024. 
\item Formal Methods: Theory and Practice \textit{(invited panel discussion)}. Ljubljana, Slovenia. PLMW at ICFP 2022.
\item Challenges Scaling Declarative Program Analysis. University of Illinois at Chicago. 2022.
\item Declarative Program Analysis at Scale. Syracuse University. 2022.
\item Tunable Abstract Abstract Machines. Workshop on Declarative Program Analysis (co-located with PLDI). 2019.
\item Contracts for Correctness (today and tomorrow). Jet, inc. 2019.
\item The Best of Both Worlds: Tunable, Correct-by-design Static Analysis. University of Alabama at Birmingham. 2018.
\item Static Analysis with Introspective Polyvariance. Indiana University. 2016.
\item Static Analysis with Introspective Polyvariance. University of Maryland. 2016.
\end{itemize}

\section{\large Contributed (Refereed Paper) Talks} \vspace{0.2in}
\begin{itemize}
\itemsep=0.1cm
\item Load-balancing Parallel Relational Algebra. Frankfurt, Germany (remote). ISC 2020.
\item Toward Parallel CFA with Datalog, MPI, and CUDA. Oxford, UK. SW 2017.
\item Allocation Characterizes Polyvariance. Nara, Japan. ICFP 2016.
\item Pushdown Control-Flow Analysis for Free. St. Petersburg, FL. POPL 2016.
\item Partitioning 0-CFA for the GPU. Wittenberg, Germany. WFLP 2014.
\item A Unified Approach to Polyvariance in Abstract Interpretations. Alexandria, VA, USA. SW 2013.
\item A Survey of Polyvariance in Abstract Interpretations. Provo, UT, USA. TFP 2013.
\end{itemize}



\end{resume}

\center{\scriptsize{An updated publication list, personal references, and other materials are available upon request.}}
\end{document}

BB
