\documentclass[12pt]{article}
\usepackage[utf8x]{inputenc}
\usepackage[margin=0.9in]{geometry}

\usepackage{natbib}
\usepackage{color}
\definecolor{lblue}{rgb}{0,0.41,0.49}


\renewcommand{\baselinestretch}{1.1} 

\usepackage{url}
\usepackage{titling}

\setlength{\droptitle}{-1.75cm}

\begin{document}
\title{Service Statement\vspace{-0.25cm}}
\author{Thomas Gilray}
\date{}
\maketitle
\vspace{-1.45cm}


Each person brings a unique creative perspective to bear on the problems they wrestle with in life and work. Our academic community depends on a broad and diverse set of perspectives as its lifeblood.
Only by working to bring together sets of divergent viewpoints, and stepping outside views and priorities we're already steeped in, can we refine our perspectives, and the very words and categories we use, to approach deeper understanding together.
In my scholarly roles, I seek to build collaborative networks of distinct supporting perspectives to pursue research goals that cannot be achieved by just one narrower vision acting alone. In my broader professional and service roles, I invest in developing my communities so all members can participate in this kind of collaborative innovation in a sustainable manner.  

\paragraph*{Local Community}
While at UAB, I ran the annual UAB high-school programming contest (HSPC) every year and served as faculty sponsor for our student ACM/ACM-W organization, who put on this any many other community events.
Our annual programming contest brought together students from nine regional secondary schools to compete in solving a set of programming challenges. I always enjoy working with undergrads to generate a set of fun and challenging problems for these guests that encourage out-of-the-box thinking. It's always impressive to me how successfully many of the high-school-aged competitors face such difficult problems. It's beautiful how the modern accessibility of computing hardware and knowledge permits talent to shine out so unmistakably, regardless of nominal background and resum\'e line items. In my work with student ACM/ACM-W organization, I had an opportunity to help foster a positive and inclusive social environment for our students by running hackathons, leetcode nights, board-game nights, and an ``undistinguished'' lecture series where students practice their communication skills by presenting on non-research-related topics of personal interest such as archery, history of dance, blockchain technology, TempleOS, classic film, and Bollywood.

I have also mentored students outside of computer science on how to leverage programming in their research. Sometimes techniques as simple as automating data processing workflows with Python scripts and regular expressions, or SQL queries, have made a big difference for students in biology and medicine. I have frequently volunteered to speak with members of the broader public on topics related to my expertise. I've been interviewed by local press, and have appeared twice on ABC's \emph{``Talk of Alabama''} morning show\footnote{\url{https://tinyurl.com/abc-talk-of-alabama-gilray}}, once for a live interview and once for a live panel discussion with industry leaders on the impact and future of AI. 

Here at WSU I have served on qualifying exam and thesis committees, and on the hiring committee, conducting initial Zoom interviews and helping to focus our department's search for new faculty on the best possible candidates.
I am eager to hold service roles like these in the future and to be involved in building community both within my department and university. I am especially keen to develop an event similar to the HSPC here at WSU's Pullman campus.


\paragraph*{Research Community}
I have served as an NSF panelist for the Principles and Practice of Scalable Systems (PPoSS) Planning program, for the PPoSS Large program, and for the Correctness for Scientific Computing Systems (SHF CS\textsuperscript{2}) program. I have found serving on these panels to be heartening in that the process has always appeared to me as a demonstration of our best scientific ideals: a wide-ranging group of scholars, with university faculty, PIs from national labs, and industry entrepreneurs all represented, debating proposed work on its intellectual merits, planning, presentation, and broader impacts. It is a satisfying way to give back to the community, and I aim to involve myself more regularly in the future and to submit to more DOE-program solicitations and to volunteer for their panels as well.

I have also served on PCs for conferences including Transactions on Computational Biology and Bioinformatics (TCBBSI), the MiniKanren Workshop (MKW), the Scheme Workshop (SW), the Symposium on Applied Computing (SAC), the Dynamic Languages Symposium (DLS, co-located with SPLASH), the Programming Language Design and Implementation Student Research Competition (PLDI SRC), and the International Conference on Software Engineering (ICSE). I have served on the external review committee for the ACM International Conference on Principles of Programming Languages (POPL) for three years, and for the ACM Transactions on Programming Languages and Systems (TOPLAS). I have also served on the Artifact Evaluation Committee for the European Conference on Object-Oriented Programming (ECOOP). I plan to continue serving on PCs and to volunteer my time especially for reviewing at venues like POPL, PLDI, ICSE, CAV, and others directly relevant to my research focus.

In 2019, I served as program committee chair for the Scheme Workshop (SW), organizing a series of refereed and invited talks. This workshop was always special to me as the venue that first accepted my own work and gave me a chance to cut my teeth at academic writing and speaking as a new graduate student in programming languages. I met several of my long-term colleagues through the Scheme Workshop and want to continue supporting it.

I plan to further serve my research community by developing a workshop that can bring together researchers working on high-performance systems and declarative languages for reasoning and analysis. In 2019 I was invited to speak at the Workshop on Declarative Program Analysis (DPA), which was a wonderful event in this space. I met the founder of Semmle (later aquired by Microsoft's GitHub) and had some fascinating discussions with him and a few others. Unfortunately, this workshop has not been held since and I would like to either pick up the baton on running DPA or start a similar event with a mission to bring together researchers working on high-performance declarative \emph{reasoning} systems more broadly.


\end{document}
