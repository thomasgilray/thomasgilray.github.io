% LaTeX file for resume / cv 
% This file uses the resume document class (res.cls)


\documentclass[line]{res} 
\usepackage{color}
\usepackage[left=0.35cm,right=2.95cm,top=2.15cm]{geometry}
\usepackage{multicol}
\usepackage{hyperref}

\usepackage{enumitem}

%\setlength{\textheight}{9.5in} % increase text height to fit on 1-page 

\newcounter{problemcounter}
\newcommand{\paper}{%
  \stepcounter{problemcounter}%
  \textbf{\theproblemcounter}}

\begin{document}
\centering
\textbf{Summary of Accomplishments}

\vspace{-0.25cm}
Thomas Gilray \\
Associate Professor, EECS
\vspace{0.25cm}

\textbf{Research Contributions}

\begin{itemize}

\item Published ``\textit{Configurable Algorithms for All-to-all Collectives.}'' at ISC 2024 (29\% acceptance). \url{https://doi.org/10.23919/ISC.2024.10528936}. This paper presents a mathematical foundation for Bruck-algorithm-based all-to-all communication and develops a variant of two-layer Bruck with a tunable radix, observing an optimal radix value empirically and analytically. 

\item Published ``\textit{Optimizing Datalog for the GPU.}'' at ASPLOS 2025 (12.7\% acceptance). \url{https://doi.org/10.1145/3669940.3707274}. This paper presents the first GPU-based Datalog engine to beat a state-of-art CPU solver, beating Souffl\'e by up to 45× (NVIDIA H100 GPUlog vs. EPYC 7543P Soufflé). A key innovation is the HISA data structure permitting fast iterated joins, in memory on the device.

\item Published ``\textit{Column-Oriented Datalog on the GPU.}'' at AAAI 2025 (23.4\% acceptance). This paper presents a column-oriented storage format for GPU-based joins. This approach shows significant improvements against leading CPU-based systems Souffl\'e and RDFox at 64 threads and against prior GPU approaches GPUjoin and HISA (our own).
  
\item Published ``\textit{Datalog with First-class Facts.}'' at VLDB 2025 (co-authored with PhD advisee Sowmith Kunapaneni). This paper presents a theory for uniqueness quantification in bottom-up logic programming as modeling fact identity and develops it into a high-performance system for manipulating (automatically indexed) structured data. 
  
\end{itemize}

\textbf{Teaching and Mentoring Accomplishments}

\begin{itemize}
  \item Developed and taught CptS 452 ``\emph{Compilers}'' which covers compilers and language runtimes in a hands-on way through a series of incremental projects where students develop an end-to-end compiler for a functional programming language. The class also covers implementing the call stack, register allocation, garbage collection, and (incidentally, via the class project) test-driven development.
  \item Currently teaching CptS 580 ``Advanced Programming Languages'' which teaches students the theory of programming languages, the lambda calculus, several related types of formal semantics, and leverages this background to teach logic programming, type systems, and program analysis, among other topics.
  \item Submitted ``\emph{Parallel Prefix Joins for Linear Recursive Datalog Rules}'' to ICS 2025 with PhD advisee Akash Rao (here at WSU). I've been working with Akash since joining WSU this fall.
  \item Submitted ``\emph{Multi-Node Multi-GPU Datalog}'' to ICS 2025 with PhD mentee Ahmedur Rahman Shovon (at UIC). I've been working with Shovon and his advisor for several years.
  \item The four papers highlit in the previous section are all collaborations with PhD mentees and advisees Ke Fan (at UIC), Ahmedur Rahman Shovon (at UIC), Yihao Sun (at SU), Arash Sahebolamri (at SU), and Sowmith Kunapaneni (here at WSU).
  \item I've been mentoring WSU BS and MS students Hunter Smith, Ralph Lewis in a project on implementing and analyzing effect handlers via delimited continuations in a functional compiler, and Zachary Werle in a project on optimization of tunable program analysis in Datalog.
\end{itemize}

\textbf{Service Accomplishments}

\begin{itemize}
    \item NSF Panelist, PPoSS Program, 2024.
    \item ARPA-H Award: HealthyDocs: Secure Interoperable Health Data Management. 2024. (My lab's share: \$1.29M)
    \item NSF Panelist, SHF CS2 Program, 2024.
    \item PC member, ICSE 2025.
    \item QE for Anais Barja, Qualifying Exam Committee Chair, WSU EECS, 2025.
    \item Faculty Search Committee, WSU EECS, 2024.
\end{itemize}


\end{document}
