% LaTeX file for resume / cv 
% This file uses the resume document class (res.cls)


\documentclass[line]{res} 
\usepackage{color}
\usepackage{randtext}
\usepackage{multicol}
\usepackage{hyperref}

\usepackage{enumitem}

%\setlength{\textheight}{9.5in} % increase text height to fit on 1-page 

\newcounter{problemcounter}
\newcommand{\paper}{%
  \stepcounter{problemcounter}%
  \textbf{\theproblemcounter}}

\begin{document}

\name{Thomas Gilray}  

\address{\bf gilray.org
\ \ \ \ \ \ \ \ \randomize{thomas.gilray@wsu.edu}}
                                  
\begin{resume}        

\vspace{-0.1in}  
I am currently an Associate Professor in Computer Science at Washington State University, and was previously Assistant Professor at the University of Alabama at Birmingham and a Victor Basili Fellow at the University of Maryland, at College Park.

%My research interests center around the design and (scalable) implementation of high-level programming languages and systems for reasoning automatically about programs. I have contributed to the design of tunable whole-program control-flow analyses, information-flow analyses, contract verification systems, and have invented novel Datalog-based languages for implementing these analyses efficiently---and declaratively. I have also contributed to high-performance-computing techniques for accelerating data-parallel relational algebra and sparse linear algebra on supercomputers.

\section{\large Education}          
    \vspace{-0.1cm}	
    \begin{tabbing}
    \hspace{2.25in}\= \hspace{2.25in}\= \kill % set up two tab positions
    {\bf University of Utah} \>Ph.D.     \>2017
    \end{tabbing}\vspace{-0.5cm}
    Developed a unified methodology for polyvariant (\textit{e.g.}, flow/call/arg/obj sensitive) program analysis.% and applied the approach to produce a 
    %specific adaptive style of polyvariance, obtaining a form of perfect precision at no asymptotic cost to analysis complexity. 
    
    I applied this framework to produce a self-reflective form of polyvariance for continuations that avoids all return-flow conflation of values (a long-standing problem for control-flow analyses), guaranteeing ideal stack precision at no cost to average or worst-case model complexity, and at in terms of human labor to implement ($\sim{}\!\!1$LOC in proposed framework).
    %
    Proved that the precision is equal to an incomputable analysis with an unbounded stack and mechanically verified the proof using the Coq proof assistant.
    %
    My dissertation is titled \textit{``Introspective Polyvariance for Control-Flow Analyses''}.
    %
   \vspace{-0.2cm} 
    \begin{tabbing}
    \hspace{2.25in}\= \hspace{2.25in}\= \kill % set up two tab positions
    {\bf University of Utah} \>M.S.     \>2012
    \end{tabbing}\vspace{-0.5cm}
%    Worked as a department TA and then as an RA in Matt Might's U-combinator lab.
    %
    \vspace{-0.2cm} 
    \begin{tabbing}
    \hspace{2.25in}\= \hspace{2.25in}\= \kill % set up two tab positions
    {\bf University of Oregon} \>B.S.     \>2010
    \end{tabbing}\vspace{-0.5cm}
%    Received a B.S. in Computer Science, Spring 2010, with a minor in Business Administration.
    %
%    Focused undergraduate studies on programming languages, interpreters, compilers, and data structures.

\vspace{0.1in}
\section{\large Employment}
   \vspace{-0.05in}  
   \begin{tabbing}
    \hspace{2.25in}\= \hspace{2.25in}\= \kill % set up two tab positions
    {\bf Associate Professor} \>Washington State University  \>2024 - \textit{Present}
   \end{tabbing}\vspace{-0.5cm}      % suppress blank line after tabbing
   Working on research into scalable, tunable program analysis, program verification, logic solvers on high performance computers and clusters, and linguistic mechanisms for enforcing correctness, security/privacy, and termination properties of software; regularly teaching undergraduate and graduate-level classes in compilers and programming languages. Won an ARPA-H subaward to investigate private and secure querying for EHR systems, contracting with with Galois, inc.
   %
   \vspace{-0.2cm}  
   \begin{tabbing}
    \hspace{2.25in}\= \hspace{2.25in}\= \kill % set up two tab positions
    {\bf Assistant Professor} \>U. of Alabama, Birmingham  \>2018 - 2024
   \end{tabbing}\vspace{-0.5cm}      % suppress blank line after tabbing
   Developed my research program in high-performance reasoning and program analysis; regularly taught undergraduate and graduate-level classes in automata theory, programming languages, and automated reasoning. Won NSF PPoSS Large, NSF PPoSS Planning, and DARPA VSPELLS grants, bringing a total of \$3.12M in new research money to UAB.
   %
   \vspace{-0.2cm}  
   \begin{tabbing}
    \hspace{2.25in}\= \hspace{2.25in}\= \kill % set up two tab positions
    {\bf Victor Basili Fellow} \>U. of Maryland, College Park  \>2016 - 2018
   \end{tabbing}\vspace{-0.5cm}      % suppress blank line after tabbing
   Joined UMD's PLUM lab with Michael W. Hicks, Jeffery Foster, and David Van Horn; worked on various collaborative projects including: soft contract verification, approximating permission-use provenance in Android, accelerating flow analyses in Datalog, and verification of faceted programs, among others. A departmental fellowship granted me great freedom to pursue long-term research.
   %
   \vspace{-0.2cm}  
   \begin{tabbing}
    \hspace{2.25in}\= \hspace{2.25in}\= \kill % set up two tab positions
    {\bf Instructor} \>U. of Maryland, College Park  \>2017 - 2018
   \end{tabbing}\vspace{-0.5cm}      % suppress blank line after tabbing
   While at UMD, I taught a section of the Intro to Programming Languages course and developed a new Compilers course in which my students built an R\textsuperscript{7}RS Scheme Compiler from scratch. 
   %
   \vspace{-0.2cm}
   \begin{tabbing}
    \hspace{2.25in}\= \hspace{2.25in}\= \kill % set up two tab positions
    {\bf Analysis Developer} \>HP, inc. (was: Fortify, inc.)     \>2013 - 2015
   \end{tabbing}\vspace{-0.5cm}      % suppress blank line after tabbing
    %
   \vspace{-0.2cm}  
   \begin{tabbing}
    \hspace{2.25in}\= \hspace{2.25in}\= \kill % set up two tab positions
    {\bf Research Assistant} \>U. of Utah     \>2011 - 2016
   \end{tabbing}\vspace{-0.5cm}      % suppress blank line after tabbing

\section{\large Teaching} \vspace{0.2in}
    
\textbf{(WSU) CptS 580: Advanced Programming Languages} (Spring 2025) \\
\textbf{(WSU) CptS 452: Compilers} (Fall 2024) \\
\textbf{(UAB) CS 660/760: Artificial Intelligence} (Fall 2019--2021, 2023) \\
\textbf{(UAB) CS 401/501: Programming Languages} (Spring 2019--2024) \\ 
\textbf{(UAB) CS 350/550: Automata and Formal Languages} (Fall 2018--2023) \\
\textbf{(UMD) CMSC 330: Intro to Programming Languages} (Spring 2018) \\
\textbf{(UMD) CMSC 430: Compilers} (Fall 2017)

\section{\large Service and Professional Development} \vspace{0.15in}
      ACM and SIGPLAN member;
      PC member, ICSE 2025;
      Faculty Search Committee, WSU EECS, 2024;
      NSF Panelist, SHF CS2 Program, 2024;
      NSF Panelist, PPoSS Program, 2024;
      External reviewer, TOPLAS 2023;
    %  PC member, PLDI Student Research Competition (SRC) 2022
      Curriculum committee member 2018-2022, UAB CS Department;
      PC member, Dynamic Languages Symposium (DLS) 2021 co-located with SPLASH;
      Co-organizer, UAB HSPC 2019-2021 (High School Programming Contest);
      PC member, Symposium On Applied Computing (SAC) 2021;
      NSF Panelist, PPoSS Program, 2021;
      External reviewer, POPL 2020;
      PC chair, Scheme and Functional Programming Workshop, 2019;
      PC member, IEEE TCBBSI 2019;
      PC member, MiniKanren Workshop, 2019;
    %  Faculty advisor to the UAB ACM \& ACM-W student chapter, 2018-present.
      Lead a student team in developing an automated grading system for UAB in 2019;
      External reviewer, POPL 2019;
      External reviewer, POPL 2018 

%\section{\large Awards \& Recognition} \vspace{0.2in}
%\begin{itemize}
\item Won ISC Hans Meuer Best Paper award (2020).
\item Featured by ALCF: collaboration with Sidharth Kumar advancing scalable MPI-based relational algebra was featured by ALCF's yearly \emph{Science Report} magazine as a research highlight growing from our 2019 D.D. grant of hours on ALCF's Theta supercomputer. \url{https://www.alcf.anl.gov/sites/default/files/2021-04/ALCF_2020ScienceReport.pdf} (page 37) (2020).
\item Invited to the Journal of Functional Programming (2019).
\item Won HiPC Best Paper award (2019).
\item DOE Directors Direction (D.D.) Grant of 2M hours on ALCF's Theta (2019).
\item Invited to the Journal of Functional Programming (2018).
\item Won PRACE ISC best paper award (2016).
\item Victor Basili Fellowship at the University of Maryland, College Park (2016).
\item Won TFP best student paper award (2013).
\end{itemize}


%\section{\large Mentoring \& Advising}

%\paragraph{MS/PhD Advisees (As Committee Chair)}
\begin{itemize}
\vspace{0.15cm}\item 2022--present Nick Netterville (MS)
\item 2022--present Ashraful Islam (PhD)
\item 2019--2020 Kyle Headley (PhD)---Left for personal reasons at the start of the pandemic.
\end{itemize}
\paragraph{MS/PhD Advisees (As Committee Member)}
\begin{itemize}
\vspace{0.15cm}\item 2021--present Akmedur Rahman Shovon (PhD)
\item 2020--present Yihao Sun (PhD)---At Syracuse University.
\item 2019--present Arash Sahebolamri (PhD)---At Syracuse University.
\item 2019--present Ke Fan (PhD)
\end{itemize}
\paragraph{Other Mentees}
\begin{itemize}
\vspace{0.15cm}\item 2022--present Landon Dyken (PhD)
\item 2022--present Akshar Patel (MS)
\item 2022--present Michael Gathara (MS)
\item 2021 Laura Thompson (BS)
\item 2018--2021 Clark Ren (MS)---Now at Amazon, inc.
\item 2016--2019 Phúc C. Nguyễn (PhD)---Now at Google, inc.
\item 2016--2018 David Darais (PhD)---Now at Galois, inc.
\item 2016--2018 Kristopher Micinski (PhD)---Now a faculty member at Syracuse University.
\item 2016--2017 Javran Cheng (MS)---Now at Google, inc.
\item 2014--2016 Guannan Wei (MS)---Now a PhD student at Purdue University.
\item 2013--2014 Maria Jenkins (BS)---Now at Pixio, inc.
\item 2011--2014 Leif Andersen (BS)---Now a PhD student at Northeastern University.
\end{itemize}


%% \section{\large Papers Currently Under Review}
%% \indent $\dagger$ Indicates a student advisee and $\ddagger$ indicates a student mentee.

%% \vspace{-0.2cm}
%% \paper. \textit{Parallel Prefix Joins for Linear Recursive Datalog Rules.}
Akash Rao$\dagger$, \textbf{Thomas Gilray}, and Ananth Kalyanaraman.
International Conference on Supercomputing.
\\(ICS) Feb 2025. (In submission)
\\ \vspace{-0.1cm}\\
\paper. \textit{Multi-Node Mutli-GPU Datalog.}
Ahmedur Rahman Shovon$\ddagger$, Yihao Sun$\ddagger$, \textbf{Thomas Gilray}, Sidharth Kumar, and Kristopher Micinski.
International Conference on Supercomputing.
\\(ICS) Feb 2025. (In submission)
\\ \vspace{-0.1cm}\\

%% \vspace{-0.4cm}

\section{\large Selected Conference Papers} 
\vspace{0.1cm}
\paper. \textit{Datalog with First-class Facts.}
\textbf{Thomas Gilray}, Arash Sahebolamri$\ddagger$, Yihao Sun$\ddagger$, Sowmith Kunapaneni$\dagger$, Sidharth Kumar, and Kristopher Micinski.
International Conference on Very Large Data Bases.
\\(VLDB) Sep 2025.
\\ \vspace{-0.1cm}\\
\paper. \textit{Multi-Node Mutli-GPU Datalog.}
Ahmedur Rahman Shovon$\ddagger$, Yihao Sun$\ddagger$, \textbf{Thomas Gilray}, Sidharth Kumar, and Kristopher Micinski.
International Conference on Supercomputing.
\\(ICS) Jun 2025.
\\ \vspace{-0.1cm}\\
\paper. \textit{Optimizing Datalog for the GPU.}
Yihao Sun$\ddagger$, Ahmedur Rahman Shovon$\ddagger$, \textbf{Thomas Gilray}, Sidharth Kumar, and Kristopher Micinski.
International Conference on Architectural Support for Programming Languages and Operating Systems.
\\(ASPLOS---12.7\% acceptance) Mar 2025. \url{https://doi.org/10.1145/3669940.3707274}
\\ \vspace{-0.1cm}\\
\paper. \textit{Column-Oriented Datalog on the GPU.}
Yihao Sun$\ddagger$, Sidharth Kumar, \textbf{Thomas Gilray}, and Kristopher Micinski.
AAAI Conference on Artificial Intelligence.
\\(AAAI---23.4\% acceptance) Feb 2025.
\\ \vspace{-0.1cm}\\
\paper. \textit{Optimizing the Bruck Algorithm for Non-uniform All-to-all Communication.}
Ke Fan$\ddagger$, \textbf{Thomas Gilray}, Valerio Pascucci, Xuan Huang, Kristopher Micinski, and Sidharth Kumar.
International ACM Symposium on High-Performance Parallel and Distributed Computing.
\\(HPDC---19\% acceptance) Jun 2022. \url{https://doi.org/10.1145/3502181.35314}
\\ \vspace{-0.1cm}\\
\paper. \textit{Load-balancing Parallel Relational Algebra.}
Sidharth Kumar and \textbf{Thomas Gilray}.
ISC High Performance.
\\(ISC---31\% acceptance) Jun 2020. \url{https://doi.org/10.1007/978-3-030-50743-5_15}
\\\textbf{Won ISC Hans Meuer Best Paper award.} \\ \vspace{-0.1cm}\\
\paper. \textit{Distributed Relational Algebra at Scale.}
Sidharth Kumar and \textbf{Thomas Gilray}.
International Conference on High Performance Computing, Data, and Analytics.
\\(HiPC---23\% acceptance) Dec 2019. \url{https://doi.org/10.1109/HiPC.2019.00014}
\\\textbf{Won HiPC Best Paper award.} \\ \vspace{-0.1cm}\\
\paper. \textit{Size-Change Termination as a Contract.}
Phúc C. Nguyễn$\ddagger$, \textbf{Thomas Gilray}, Sam Tobin-Hochstadt, and David Van Horn.
Programming Language Design and Implementation.
\\(PLDI---27\% acceptance) Jun 2019. \url{https://doi.org/10.1145/3314221.3314643}
\\\textbf{Invited to the Journal of Functional Programming.} \\ \vspace{-0.1cm}\\
\paper. \textit{Soft Contract Verification for Higher-order Stateful Programs.}
Phúc C. Nguyễn$\ddagger$, \textbf{Thomas Gilray}, Sam Tobin-Hochstadt, and David Van Horn.
Symposium on Principles of Programming Languages.
\\(POPL---23\% acceptance) Jan 2018. \url{https://doi.org/10.1145/3158139}
\\ \vspace{-0.1cm}\\
\paper. \textit{Allocation Characterizes Polyvariance.}
\textbf{Thomas Gilray}, Michael D. Adams, and Matthew Might.
International Conference on Functional Programming.
\\(ICFP---31\% acceptance) Sep 2016. \url{https://doi.org/https://doi.org/10.1145/3022670.2951936}
\\ \vspace{-0.1cm}\\
\paper. \textit{Pushdown Control-Flow Analysis for Free.}
\textbf{Thomas Gilray}, Steven Lyde, Michael D. Adams, Matthew Might, and David Van Horn.
Symposium on Principles of Programming Languages.
\\(POPL---23\% acceptance) Jan 2016. \url{https://doi.org/10.1145/2837614.2837631}
\\ \vspace{-0.1cm}\\
\paper. \textit{Dynamic Sparse-Matrix Allocation on GPUs.}
James King, \textbf{Thomas Gilray}, Robert M. Kirby, and Matthew Might.
ISC High Performance.
\\(ISC) 2016. \url{https://doi.org/10.1007/978-3-319-41321-1_4}
\\\textbf{Won PRACE ISC best paper award.} \\ \vspace{-0.1cm}\\

\vspace{-0.4cm}

%% \section{\large Journal Papers} \vspace{0.3cm}
%% \paper. \textit{Abstract Allocation as a Unified Approach to Polyvariance in Control-flow Analyses.}
\textbf{Thomas Gilray}, Michael D. Adams, and Matthew Might.
Journal of Functional Programming.
\\(JFP) 2018. 
\\\textbf{Invited to the Journal of Functional Programming.} \\ \vspace{-0.1cm}\\

%% \vspace{-0.4cm}

%% \section{\large Doctoral Theses} \vspace{0.3cm}
%% \paper. \textit{Introspective Polyvariance for Control-Flow Analyses.}
\textbf{Thomas Gilray}.
University of Utah.
\\(U of U) 2016.
\\ \vspace{-0.1cm}\\

%% \vspace{-0.4cm}

\section{\large Invited Talks} \vspace{0.2in}
\begin{itemize}
\item Challenges in High-performance Deductive Programming. AP2S: Automated Program and Proof Synthesis. Vancouver, BC. AAAI Bridge. 2024. 
\item Formal Methods: Theory and Practice \textit{(invited panel discussion)}. Ljubljana, Slovenia. PLMW at ICFP 2022.
\item Challenges Scaling Declarative Program Analysis. University of Illinois at Chicago. 2022.
\item Declarative Program Analysis at Scale. Syracuse University. 2022.
\item Contracts for Correctness (today and tomorrow). Jet, inc. 2019.
\item The Best of Both Worlds: Tunable, Correct-by-design Static Analysis. University of Alabama at Birmingham. 2018.
\item Static Analysis with Introspective Polyvariance. Indiana University. 2016.
\item Static Analysis with Introspective Polyvariance. University of Maryland. 2016.
\end{itemize}

\section{\large Contributed Talks} \vspace{0.2in}
\begin{itemize}
\itemsep=0.1cm
\item Load-balancing Parallel Relational Algebra. Frankfurt, Germany (remote). ISC 2020.
\item Toward Parallel CFA with Datalog, MPI, and CUDA. Oxford, UK. SW 2017.
\item Allocation Characterizes Polyvariance. Nara, Japan. ICFP 2016.
\item Pushdown Control-Flow Analysis for Free. St. Petersburg, FL. POPL 2016.
\item Partitioning 0-CFA for the GPU. Wittenberg, Germany. WFLP 2014.
\item A Unified Approach to Polyvariance in Abstract Interpretations. Alexandria, VA, USA. SW 2013.
\item A Survey of Polyvariance in Abstract Interpretations. Provo, UT, USA. TFP 2013.
\end{itemize}



\end{resume}

\end{document}
